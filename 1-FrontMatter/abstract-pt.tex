%!TEX root = ../template.tex
%%%%%%%%%%%%%%%%%%%%%%%%%%%%%%%%%%%%%%%%%%%%%%%%%%%%%%%%%%%%%%%%%%%%
%% abstract-pt.tex
%% NOVA thesis document file
%%
%% Abstract in Portuguese
%%%%%%%%%%%%%%%%%%%%%%%%%%%%%%%%%%%%%%%%%%%%%%%%%%%%%%%%%%%%%%%%%%%%

\typeout{NT FILE abstract-pt.tex}%

A computação em nuvem tornou-se um elemento central dos sistemas de software modernos, oferecendo às organizações uma flexibilidade, escalabilidade e eficiência de custos sem precedentes. À medida que as empresas transitam cada vez mais de infraestruturas locais para arquiteturas nativas da nuvem (\emph{cloud-native}), as aplicações \emph{desktop} tradicionais enfrentam limitações crescentes em termos de acessibilidade e integração com serviços distribuídos. As soluções baseadas na \emph{web}, suportadas por plataformas de nuvem elásticas, permitem uma disponibilidade contínua, melhores práticas de segurança e \emph{pipelines} de implementação simplificados. Esta mudança de paradigma estabeleceu a computação em nuvem não apenas como uma tendência tecnológica, mas como um modelo padrão para o fornecimento de software fiável e centrado no utilizador em larga escala.

Esta tese apresenta o desenho e a migração do \sysname, uma aplicação \emph{desktop} em Java 8 originalmente construída para ambientes de rede local, para uma aplicação \emph{web} moderna alojada na nuvem. O \sysname{} é uma plataforma responsável pela emissão de documentos de faturação, geração e comunicação de ficheiros \emph{Standard Audit File for Tax} (SAF-T PT) à Autoridade Tributária, submissão de declarações de Imposto do Selo e integração de dados de faturação com sistemas bancários e de Gestão de Relacionamento com o Cliente através de \emph{web services}. O trabalho inclui a definição de uma arquitetura baseada na nuvem escalável e segura, o redesenho do sistema numa aplicação \emph{web} modular e a implementação de uma prova de conceito que valida a abordagem proposta. Esta migração aborda os desafios de adaptar uma solução baseada em \emph{desktop} e de implementação local a um ambiente distribuído e orientado a serviços, preservando simultaneamente a robustez e a fiabilidade exigidas pelo seu contexto operacional. A solução resultante demonstra como aplicações \emph{desktop} locais podem ser transformadas com sucesso para satisfazer as exigências dos ecossistemas modernos habilitados pela nuvem.

% Palavras-chave (European Portuguese Keywords)
\keywords{
  Computação em Nuvem \and
  Migração de Aplicações Web \and
  Arquitetura de Software \and
  Serviços na Nuvem
}
% to add an extra black line
